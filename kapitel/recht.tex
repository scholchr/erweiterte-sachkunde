%!TEX root = ../erweiterte_sachkunde.tex
\chapter{Recht}


\section{Überblick}

    \subsection{Überblick Gesetze}
    \begin{itemize}
        \item Tierschutzgesetz (bundesweit, regelt Tötung etc.)
        \item BGB
        \item Landeshundegesetz NRW
        \item Tierschutzhundeverordnung (bundesweit)
        \item Landesforstgesetz NRW (Hund im Wald)
        \item Hundesteuersatzung (städtespezifisch)
        \item Ordnungsbehördengesetz (selten)
        \item Ordnungsverordnung der Stadt Duisburg
        \item Mietvertrag
    \end{itemize}

    \clearpage
    \subsection{Hundespezifische Regelungen}
    \begin{itemize}
        \item §3 LHG Gefährliche Hunde
        \begin{itemize}
            \item Pitbull Terrier
            \item American Staffordshire Terrier
            \item Staffordshire Bullterrier
            \item Bullterrier
            \item im Einzelfall gefährliche Hunde
        \end{itemize}
        \item §10 LHG Hunde bestimmter Rassen:
        \begin{itemize}
            \item Alano
            \item American Bulldog
            \item Bullmastiff
            \item Mastiff
            \item Mastino Espanol
            \item Mastino Napoletano
            \item Fila Brasileiro
            \item Dogo Argentini
            \item Rottweiler
            \item Tosa Inu
        \end{itemize}
        \item §11 LHG Große Hunde (sog. 20/40er)
        \begin{itemize}
            \item min. 40cm Widerristhöhe
            \item \textbf{oder} min. 20kg Gewicht
        \end{itemize}
        \item Alle Hunde
    \end{itemize}

    \begin{tabular}{c|c}
        \textbf{Schutz Mensch vor Tier} & \textbf{Schutz Tier vor Mensch} \\
        \hline
        LHG & Tierschutzgesetz \\
        OBG & Tierschutzhundeverordnung \\
    \end{tabular}


\clearpage
\section{Regelung für alle Hunde}
    \begin{itemize}
        \item Alle Hunde sind so zu halten, zu führen und zu beaufsichtigen, dass von Ihnen keine Gefahr für Leben und Gesundheit von Menschen und Tieren ausgeht.
        \item \textbf{Leinenpflicht LHG}:
        \begin{itemize}
             \item Fußgängerzonen, Haupteinkaufsbereiche und andere innerörtliche Bereiche, Straße und Plätze mit vergl. Publikumsverkehr
             \item In der Allgemeinheit zugänglichen, umfriedeten\footnote{\glqq umzäunt\grqq{}, auch durch niedrige Hecke etc.} Park, Garten- und Grünanlagen, einschl. Kinderspielplätzen mit Ausnahme besonders ausgewiesener Hundeauslaufflächen (7 in DU)
             \item bei öffentlichen Versammlungen, Aufzügen, Volksfesten und sonstigen Veranstaltungen mit Menschenansammlungen
             \item in öffentlichen Gebäuden, Schulen und Kindergärten
         \end{itemize}
         \item \textbf{Leinenpflicht Landesforstgesetz}:
         \begin{itemize}
             \item Im Wald (gilt nicht für Jagdhunde bei der Jagd und Polizeihunde)
             \item Ausnahme: Auf Wegen mit \glqq Kontrolle\grqq{} über den Hund: Keine Anleinpflicht
             \item Reiterwege: Betretungsverbot
         \end{itemize}
         \item \textbf{Sonderfall Landschaftsgesetz} bzgl. Landschaftsschutzgebieten:
         \begin{itemize}
             \item Es ist verboten wildlebende Tiere mutwillig zu beunruhigen oder ohne vernünftigen Grund zu fangen, zu verletzen oder zu töten.
             \item Keine grundsätzliche Anleinpflicht, aber Hausrecht der Schäfer und Bauern
         \end{itemize}
         \item \textbf{Probleme in Landschaftsschutzgebieten}:
         \begin{itemize}
             \item Jagen und Reißen von Schafen und Wildtieren
             \item Verunreinigung durch Hundekot (Schafe fressen nicht, Heu wird nicht abgenommen)
             \item Löcher vom Buddeln
             \item Beschädigung von Zäunen und Gattern
         \end{itemize}
         \item \textbf{Ordnungsverordnung Duisburg}:
         \begin{itemize}
             \item Leinenpflicht: ausgewiesene Park, Garten- und Grünanlagen
             \item Betretungsverbot: Kinderspielplätze, Sandspielflächen, Liegewiesen und Sportflächen
             \item Weitere Pflichten: Auf Verkehrsflächen und Grünanlagen dürfen weder Personen noch Tiere gefährdet oder Sachen (insbes. Gehwege, Plätze und Blumenanlagen) beschmutzt oder beschädigt werden.
         \end{itemize}

         \clearpage
         \item \textbf{Besondere Regelungen}:
         \begin{itemize}
             \item Betretungsverbot auf Wochenmärkten, Friedhöfen
             \item Naturschutzgebiet: Leinenpflicht
             \item Privatgrundstücke: Regelung des Hauseigentümers
             \item VRR: Leinenpflicht und bei Gefährdung von Personen Maulkorbpflicht
         \end{itemize}
    \end{itemize}


\section{Zusätzliche Regelung für große Hunde}
    \begin{itemize}
        \item Anzeige bei Behörden
        \item Mikrochip
        \item Haftpflichtversicherung
        \item Sachkunde und Zuverlässigkeit des Halters
        \item Leinenpflicht in bebauten Ortschaften
    \end{itemize}


\section{Zusätzliche Regelung für gefährliche Hunde nach §3 LHG}
    \begin{itemize}
        \item Zuchtverbot
        \item Haltung: setzt besonderes privates oder öffentlichen Interesses voraus, zusätzlich:
        \begin{itemize}
            \item Volljährigkeit
            \item Sachkundenachweis durch Amtstierarzt
            \item Zuverlässigkeit durch Führungszeugnis
            \item Nachweis der ausbruchssicheren und verhaltensgerechten Unterbringung
            \item Haftpflichtversicherung mit Mindestdeckungssumme
            \item Mikrochip
        \end{itemize}
        \item Verhaltenspflicht:
        \begin{itemize}
            \item Anleinpflicht außerhalb des Grundstücks, Befreiungsmöglichkeit nach Verhaltensprüfung
            \item Maulkorbpflicht ab 6. Lebensmonat, Befreiungsmöglichkeit nach Verhaltensprüfung
            \item \glqq feste\grqq{} Hand von Halter und Aufsichtspersonen
            \item Sachkunde, Zuverlässigkeit und Volljährigkeit auch für Aufsichtspersonen
            \item Verbot mehrere gefährliche Hunde gleichzeitig zu führen
            \item Mitteilungspflichten (Umzug)
        \end{itemize}
        \item Hundesteuer Duisburg: keine Sonderregelung für §3 Hunde, Steuerbefreiung für ALG II Empfänger
    \end{itemize}

 \section{\glqq Erleichterungen\grqq{} bei §10 LHG Hunden gegenüber §3 Hunden}
    \begin{itemize}
           \item Kein Zuchtverbot
           \item Kein besonderes Interesse für neue Haltung erforderlich
           \item Sachkundeprüfung und Verhaltensprüfung muss nicht zwingend durch Amtstierarzt erfolgen
       \end{itemize}


\section{Gefährliche Hunde im Einzelfall}
    \begin{itemize}
        \item Hunde, die einen Menschen gebissen haben (Ausnahme: Verteidigung einer strafbaren Handlung)
        \item Hunde, die einen Menschen in Gefahr drohender Weise angesprungen haben
        \item Hunde, die einen anderen Hund gebissen haben und dies ohne Angriff oder trotz Unterwerfung des anderen Hundes geschehen ist
        \item $\rightarrow$ mögliche Einschläferung
        \item $\rightarrow$ einmal gefährlicher Hund = immer gefährlicher Hund
        \item $\rightarrow$ keine Möglichkeit der Befreiung von Anlein- oder Maulkorbpflicht
    \end{itemize}


\section{Tierschutzgesetz, Tierschutzhundeverordnung, BGB, GG}
    \subsection{Tierschutzgesetz}
        \begin{itemize}
            \item Grundsatz: Kein unnötigen Schmerzen, Leid oder Schäden
            \item Tierhaltung:
            \begin{itemize}
                \item Art- und bedarfsgerechte Ernährung, Pflege und verhaltensgerechte Unterbringung
                \item artgemäße Bewegung
            \end{itemize}
            \item Tötung von Wirbeltieren: Nur unter Betäubung oder unter Vermeidung von Schmerzen. Kenntnisse und Fähigkeiten notwendig.
            \item behördliche Erlaubnis bei:
            \begin{itemize}
                \item Tieren im TH oder Zoologischen Garten
                \item Wirbeltierhandel
                \item Zurschaustellung
                \item Wirbeltierschädlingsbekämpfung
            \end{itemize}

            \item Maßnahmen bei Verstössen:
            \begin{itemize}
                \item Vorführung beim Tierarzt
                \item Vorführung beim Amtsveterinär
                \item Anordnung nur eine bestimmte Anzahl an Tieren zu halten
                \item Sicherstellung
            \end{itemize}
        \end{itemize}

    \subsection{Tierschutzhundeverordnung}
        \begin{itemize}
            \item Überprüfung nach Art der Beschwerde (Muskulatur und Krallenlänge, Ernährungszustand, Gesundheitszustand)
            \item Ausstellungsverbot amputierter Hunde
            \item Halten im Freien:
            \begin{itemize}
                \item Schutzhütte: Innenraum mit Körperwärme warmhaltbar
                \item Liegeplatz: witterungsgeschützt, schattig, wärmegedämmter Boden
            \end{itemize}
            \item Regelung zur Zwingerhaltung
            \item Anforderung an Anbindehaltung
            \item Fütterung und Pflege
            \item Welpentrennung erst nach 8 Wochen
            \item Züchtung: 1 Person pro 10 Zuchthunde und ihre Welpen
        \end{itemize}

    \subsection{BGB}
        \begin{itemize}
            \item Tiere sind keine Sachen
            \item besondere Gesetze für Tiere
            \item Soweit nicht anders geregelt: für Sachen geltende Vorschriften anwendbar
            \item Findeltiere: Anzeige- und Ablieferungspflicht bei Behörde
        \end{itemize}

    \subsection{Grundgesetz}

        Tierschutz ist Staatsziel
