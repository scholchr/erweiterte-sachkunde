%!TEX root = ../erweiterte_sachkunde.tex
\chapter{Hundeverhalten III}


\section{Aggressionsverhalten}
    \subsection{Was ist Aggression?}
    Gehört zum \textbf{agonistischen Verhalten}:
    \begin{itemize}
        \item Verhaltensreaktion auf spezif. Umweltreize/Bedrohung
        \item Verhaltesnreaktion in einem Konflikt um Ressource
    \end{itemize}

    \subsection{Äußerung agonistischen Verhaltens - 4F's}
    \begin{itemize}
        \item Freeze (Erstarren)
        \item Flirt (Übersprungshandlung)
        \item Flight (Flucht)
        \item Fight (Kampf)
    \end{itemize}

    \subsection{Aggression beim Hund}
    \begin{itemize}
        \item natürliche Strategie der Verhaltensanpassung an die Umwelt
        \item obligatorisches Regulativ für das Zusammenleben
        \item umfasst alle Verhaltensweisen, die zur Beeinträchtigung der physischen und psychischen Integrität oder Freiheit von anderen führen
        \item allgemein: Aggression kommt häufig vor, wenn:
        \begin{itemize}
            \item der Hund daran gehindert wird, etwas zu tun oder
            \item der Hund dazu gebracht werden soll, etwas zu tun, was er nicht möchte
        \end{itemize}
    \end{itemize}

    \clearpage
    Aggression = Summe von:
    \begin{itemize}
        \item \textbf{erlerntem Verhalten}, d.h. Erfahrungen während der ersten Lebenswochen und der tägliche Erfahrung des gesamten Lebens
        \item \textbf{angeborenen Eigenschaften}: genetische Disposition/Rasse
        \item dem \textbf{körperlichen Zustand} des Tieres
        \item der gesamten augenblicklichen \textbf{Situation}
    \end{itemize}

    \subsection{Ursachen und Kategorien aggressiver Verhaltensweisen}
        \begin{itemize}
            \item Angstbedingt: negative Erfahrungen oder Mangel an Erfahrungen
            \item in Verbindung mit Status/Rangfolge
            \item aus Frustration
            \item hormonell bedingt (zwischen Rüden, zwischen Hündinnen)
            \item umgerichtete Aggression\footnote{A. ist auf ein Ziel gerichtet, das Ziel verschwindet $\rightarrow$ A. wird auf ein neues Ziel gerichtet}
            \item Schmerz- und Schockbedingt
            \item Folge einer organischen Erkrankung (Schilddrüse, Tumor im Kopf, Tollwut, Zyklusstörung)
            \item sog. \glqq idiopathische\grqq Aggression (\glqq Cocker Wut\grqq)
        \end{itemize}

    \subsection{Aggression und Ressourcen}
    \begin{itemize}
        \item Erwerb/Verteidigung einzelner Objekte (Futter, Ruheplätze, Spielzeug, ggf. Mensch als Adressat = kompetitiv sozial)
        \item Unversertheit des Körpers
        \item Territorial
        \item Hormonell: Mütterliche Schutzaggression, sexuelle Rivalität
    \end{itemize}

    \subsection{Innerartliche Aggression}
        \begin{itemize}
            \item Gegenüber Artgenossen, die zum Rudel gehören
            \begin{itemize}
                \item hormonell
                \item Mütterliche Aggression
                \item Spielaggression
            \end{itemize}
            \item Gegenüber fremden Hunden
            \begin{itemize}
                \item häufige Ursache: mangelhafte Sozialisation
                \item Missverständnisse in der Kommunikation
            \end{itemize}
        \end{itemize}

    \subsection{Aggression gegen Menschen}
        \begin{itemize}
            \item gegen unbekannte Menschen durch:
            \begin{itemize}
                \item mangelhafte Sozialisation
                \item Verteidigung des Territoriums
                \item unangemessenes Jagdverhalten (Mensch als Beutetier)
            \end{itemize}
            \item gegen Besitzer oder andere Familienmmitglieder
        \end{itemize}

    \subsection{Ausdrucksformen des Agressionsverhaltens}
        \subsubsection{offensives Drohverhalten}
            \begin{itemize}
                \item Fixieren
                \item Knurren als akustische Unterstützung der optischen Drohung: tief und laut
                \item Vorn-Zähneblecken mit kurzem Mundwinkel
                \item Über-dem-Gegner-stehen
                \item Über-der-Schnauze-beißen
                \item Haarsträuben
            \end{itemize}
        \subsubsection{defensives Drohverhalten}
            \begin{itemize}
                \item Abwehrdrohen: Voll-Zähneblecken (Mundwinkel nach hinten gezogen), oft mit Fauchen und Keifen
                \item Abwehrschnappen: Bisse in die Luft
                \item Gebissklappern
                \item Abwehrbeissen aus defensiver Haltung Richtung Hals und Ohren
            \end{itemize}

    \subsection{Aggressive Auseinandersetzung}
        \begin{itemize}
            \item Drohen
            \item Imponieren
            \item Beißen:
            \begin{itemize}
                \item Angriff
                \item Beißschütteln
                \item ernsthafter Beschädigungskampf
                \item Randnotiz: Beißen Richtung Kopf: sozial \glqq geschädigt\grqq
            \end{itemize}
        \end{itemize}

    \subsection{Beißhemmung}
        \begin{itemize}
            \item erlernter Bestandteil der ritualisierten aggressiven Kommunikation
            \item verhindert Eskalation
        \end{itemize}

    \subsection{Dominanz}
        \begin{itemize}
            \item Keine Charakter- oder Wesenseigenschaft
            \item bezeichnet eine Eigenschaft von Beziehungen und beschreibt das Verhältnis zweier Individuen
        \end{itemize}

    \subsection{Einfluss des Lernens auf aggressives Verhalten beim Hund}
        \subsubsection{Positive Folgen}
            Positive Folgen verstärken aggressives Verhalten: Erfolg
            \begin{itemize}
                \item Distanz bleibt bestehen oder wird vergrößert
                \item Gegner geht weg, Ressource bleibt im Besitz
                \item Verhalten des Besitzers bestärkt den Hund z.B. durch Fehlinterpretation: \glqq Hund will mich beschützen\grqq
            \end{itemize}
        \subsubsection{Negative Folgen}
            Negative Folgen reduzieren aggressives Verhalten: kein Erfolg
            \begin{itemize}
                \item Das Verhalten hat sich nicht gelohnt
                \item Distanz wird nicht aufrechterhalten oder vergrößert
                \item Ressource bleibt nicht im Besitz bzw. wird nicht erworben
            \end{itemize}

    \subsection{Häufige Reaktionen des Menschen auf aggressives Verhalten}
        \begin{itemize}
            \item Gegenaggression = Strafe
            \item Versuch der \glqq Beruhigung\grqq
            \item Vermeiden entsprechender Situationen
            \item \textbf{Für eigene Sicherheit sorgen!}
        \end{itemize}

\section{Verhaltenstest}
    \subsection{Ziel des Verhaltenstestes}
        Erkennen von Hunden mit gestörter aggressiver Kommunikation = \textbf{übersteigertes} Aggressionsverhalten:
        \begin{itemize}
            \item biologisch und in der Ursache nicht nachvollziehbar
            \item tritt unvermittelt auf, oft ohne Ablauf der sog. Eskalationsleiter
        \end{itemize}

    \subsection{Mögliche aggressive Verhaltensmuster - Eskalationsstufen}
        \begin{enumerate}
            \item \textbf{keine aggressiven Signale}: Hund bleibt neutral oder zeigt Meideverhalten
            \item \begin{enumerate}
                    \item \textbf{akustische Signale}: Knurren/tiefes Bellen
                    \item \textbf{optische Signale}: Zähneblecken, Drohfixieren
                \end{enumerate}
            \item \textbf{Schnappen}: Beißbewegung aus einiger Entfernung mit oder ohne Drohgesten, \textbf{keine Annäherung}
            \item \textbf{Schnappen mit unvollständiger Annäherung}: Stehen bleiben in einiger Distanz
            \item \textbf{Beißen} (Beißversuche) oder \textbf{Angreifen} (Angriffsversuche: Annäherung mit hoher Geschwindigkeit und Zustoßen) + Drohgesten
            \item Ebenso, aber: \textbf{ohne mimische oder akustische Signale}
            \item \textbf{Beruhigung des Hundes erst nach 10 Minuten}
        \end{enumerate}
        5-7 bedeutet: Prüfung nicht bestanden!

    \subsection{Anforderungen an den Verhaltenstest}
        \begin{itemize}
            \item Hunde werden vielen Reizen und Situationen konfrontiert, v.a. solchen, die bekannterweise Aggressionsverhalten auslösen
            \item Bewertung jeder Reaktion nach Schema Eskalationsstufe und Multiplikatoren:
            \begin{itemize}
                \item Faktor 1: Verhalten ist nachvollziehbar
                \item Faktor 2: nachchvollziehbar, aber unerwünscht
                \item Faktor 3: gravierend und nicht mehr akzeptabel
            \end{itemize}
        \end{itemize}

    \subsection{Aufbau des Verhaltenstests}
        \subsubsection{Eingezäuntes Gelände}
            \begin{itemize}
                \item Begrüßung gesamte Gruppe
                \item Hundebegegnung
                \item Gehorsam: Einzelprüfung
                \item Spezielle Reize und Situationen Mensch-Hund
                \item Spezeille Reize und Situationen, aber ohne Mensch
            \end{itemize}
        \subsubsection{Testsituationen im Freigelände}
            \begin{itemize}
                \item Spaziergang mit der Gruppe:
                \begin{itemize}
                    \item Fahrradfahrer
                    \item Jogger
                    \item Lärmende Menschengruppe
                    \item Kinderwagen etc.
                    \item Gehorsamsübungen: bei Fuß gehen, abrufen ...
                \end{itemize}
                \item Testsituationen in belebter Stadt:
                \begin{itemize}
                    \item Gehorsamsübungen
                    \item Beobachtung aller sich ergebender Situationen
                    \item Anleinen des Hundes in Einkaufszone ohne Besitzer (unter Aufsicht des Prüfers)
                \end{itemize}
            \end{itemize}