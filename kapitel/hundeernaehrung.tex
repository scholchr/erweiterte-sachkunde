%!TEX root = ../erweiterte_sachkunde.tex
\chapter{Hundeernährung}


\section{Verdauung}

    \subsection{Verdauungstrakt des Hundes}
    \begin{itemize}
        \item Gebiss: Schneidende Funktion, wird nicht so sehr zerkleinert wie beim Menschen
        \item Magen: Großes Fassungsvermögen, Säureschutz
        \item Darm: kurz, erfordet hochverdauliches Futter
    \end{itemize}

    \subsection{Unterschiede zwischen Mensch und Hund}
    Ein Hund hat:
    \begin{itemize}
        \item kürzere Dauer der Mahlzeit
        \item z.T. größeren, dehnbareren Magen
        \item kürzeren Verdauungsprozess
        \item geringeren Bedarf an Kohlehydraten
        \item höheren Bedarf an Proteinen und Fetten
    \end{itemize}

    \subsection{Ablauf der Verdauung}
    \begin{itemize}
        \item \textbf{Zerkleinerung} durch das Gebiss
        \item \textbf{Aufschluss} durch Enzyme im Magen und Darm
        \item \textbf{Aufnahme} in Lymphe und Blut durch den Dünndarm
        \item \textbf{Umbau} in körpereigene Substanz (Muskulatur, Blutkörper, Haare, etc.) durch die Leber
    \end{itemize}


\section{Energie und Nährstoffe}

    \subsection{Hauptnährstoffe}
        Die Hauptnährstoffe werden in \textbf{Körpersubstanz} und \textbf{Energie} umgewandelt. \\
        Hauptnährstoffe sind:
        \begin{itemize}
            \item Proteine (Eiweiß)
            \item Kohlenhydrate (Stärke, Zucker)
            \item Fette
        \end{itemize}

\section{Futtermittel}

\section{Praktische Fütterung}