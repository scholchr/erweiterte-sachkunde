%!TEX root = ../erweiterte_sachkunde.tex
\chapter{Hundeernährung}


\section{Verdauung}

    \subsection{Verdauungstrakt des Hundes}
    \begin{itemize}
        \item Gebiss: Schneidende Funktion, wird nicht so sehr zerkleinert wie beim Menschen
        \item Magen: Großes Fassungsvermögen, Säureschutz
        \item Darm: kurz, erfordet hochverdauliches Futter
    \end{itemize}

    \subsection{Unterschiede zwischen Mensch und Hund}
    Ein Hund hat:
    \begin{itemize}
        \item kürzere Dauer der Mahlzeit
        \item z.T. größeren, dehnbareren Magen
        \item kürzeren Verdauungsprozess
        \item geringeren Bedarf an Kohlehydraten
        \item höheren Bedarf an Proteinen und Fetten
    \end{itemize}

    \subsection{Ablauf der Verdauung}
    \begin{itemize}
        \item \textbf{Zerkleinerung} durch das Gebiss
        \item \textbf{Aufschluss} durch Enzyme im Magen und Darm
        \item \textbf{Aufnahme} in Lymphe und Blut durch den Dünndarm
        \item \textbf{Umbau} in körpereigene Substanz (Muskulatur, Blutkörper, Haare, etc.) durch die Leber
    \end{itemize}


\section{Energie und Nährstoffe}

    \subsection{Hauptnährstoffe}
        Die Hauptnährstoffe werden in \textbf{Körpersubstanz} und \textbf{Energie} umgewandelt. \\
        Hauptnährstoffe sind:
        \begin{itemize}
            \item Proteine (Eiweiß)
            \item Kohlenhydrate (Stärke, Zucker)
            \item Fette
        \end{itemize}

        \subsubsection{Proteine}
            \begin{itemize}
                \item Funktion: Körpersubstanz, Enzyme
                \item Funktion nach Abbau: Energiegewinnung
                \item Biologische Wertigkeit:
                \begin{itemize}
                    \item Hoch (Rohprotein: Ei)
                    \item Mittel (Bindegewebe)
                    \item Niedrig (Pflanzliche Proteine)
                \end{itemize}
            \end{itemize}

        \subsubsection{Kohlenhydrate}
            \begin{itemize}
                \item Funktion: Energielieferant
                \item Speicher in Muskulatur, Leber
                \item Zucker ungeeignet (Karies)
                \item Kohlenhydratquelle meist Reis oder Mais (Kartoffeln ungeeignet)
                \item Durchfallgefahr durch zuviel ungeeignete (rohe) Stärke im Futter, da sie Wasser im Darm bindet
            \end{itemize}

        \subsubsection{Fette}
            \begin{itemize}
                \item Funktion: Energielieferant (2,5mal soviel wie Kohlenhydrate)
                \item Aufbau: gesättigte/ungesättigte Fettsäuren
                \item Wichtig für Dauerleistung
                \item Omega-3-Fettsäuren essentiell und entzündungshemmend (Fischöle)
                \item Geschmacksträger im Futter
            \end{itemize}

    \subsection{Mineralstoffe}
        \begin{itemize}
            \item Calcium, Phosphor
            \begin{itemize}
                \item Skelettaufbau
                \item Ausgewogenes Zufuhrverhältnis: 1,5 : 1
            \end{itemize}
            \item Magnesium
            \item Kalium, Natrium
            \item Spurenelemente (Eisen, Zink, Kupfer, Kobalt, Selen, Jod u.a.)
        \end{itemize}

    \subsection{Vitamine}
        \begin{itemize}
            \item Fettlösliche Vitamine:
            \begin{itemize}
                \item A, E, D, K
                \item müssen mit Fett aufgenommen werden
                \item werden im Körperfett gespeichert
                \item Überversorgung = Vergiftungsgefahr
            \end{itemize}
            \item Wasserlösliche Vitamine: B*, H, C wird selbst gebildet
        \end{itemize}


\section{Futtermittel}

    \subsection{Futtermittelarten}
        \begin{itemize}
            \item \textbf{Alleinfutter}: enthält alle Nährstoffe
            \item \textbf{Ergänzungsfuttermittel}: z.B. Mineralstoffmischung
            \item \textbf{Einzelfuttermittel}: z.B. getrockneter Pansen
        \end{itemize}

    \subsection{Zutaten/Etikettinformation}
        \begin{itemize}
            \item Chemische Futteranalyse ergibt keine Aussage über Verdaulichkeit
            \item Zutatenliste ist wichtiger als Bezeichnung
            \item Zutatenliste wird immer in mengenmäßig absteigender Reihenfolge angegebene
            \item Vorsicht bei unklaren Begriffen wie tierische/pflanzliche Nebenerzeugnisse
        \end{itemize}

    \subsection{Was macht ein gutes Hundefutter aus?}
        \begin{itemize}
            \item Deklaration:
            \begin{itemize}
                \item Keine Farbstoffe, kein Zucker
                \item Tierisches Protein an erster Stelle
                \item Fütterungsmengenangabe
            \end{itemize}
            \item Am Hund:
            \begin{itemize}
                \item Kotmenge(50g pro 100g Trockenfutter) und -beschaffenheit
                \item glänzendes Fell
            \end{itemize}
        \end{itemize}


\section{Praktische Fütterung}
    \subsubsection{Bedarfsarten}
        \begin{tabular}{c|ccc}
            \textbf{Bedarf}         &   \textbf{Energie}    & \textbf{Mineralstoffe}    & \textbf{Protein} \\
            \hline
            Trächtigkeit, Säugen    &   +++                 & +++           & ++ \\
            Aufzucht                &   +                   & +++           & ++  \\
            Leistung                &   +++                 &               & ++ \\
            Alter Hund              &   -                   & +             & + \\
                                    &                       & +Vitamin      & \\
        \end{tabular}

    \subsection{Futterallergien}
        \begin{itemize}
            \item Oft Zusatzstoffe als Ursache
            \item Meist durch Futterbestandteile (Eiweisse)
            \item Herangehensweise: \textbf{Ausschlussdiät} mit 12 Wochen \glqq Inkubationszeit \grqq{}
        \end{itemize}

    \subsection{B.A.R.F}
        \begin{itemize}
            \item Biologisch Artgerechtes Rohfutter
            \item Soll Wolfsernährung ähneln
            \item Soll bei Allergien helfen
            \item Ggf. Preisgünstiger
            \item Nachteile:
            \begin{itemize}
                \item Rationen ausgewogen?
                \item Mineralstoffmangel/-überschuss
                \item Verdaulichkeit?
                \item Salmonellengefahr
                \item Knochenfütterung
                \item Schlecht geeignet für wachsende/tragende/alte Hunde
            \end{itemize}
        \end{itemize}

    \subsection{Randnotiz: Vegetarische/Vegane Hundeernährung}
        \begin{itemize}
            \item Abzuraten, da Hunde Karnivoren (Fleischfresser) sind
            \item Hunde sind nicht in der Lage pflanzliche Nährstoffe in adäquatem Maße zu verwerten
            \item Gefahr der Mangelernährung
            \item Vegetarische/Vegane Ernährung für Welpen und trächtige Hunde ungeeignet
            \item Pflanzliche Nahrung belastet Nieren, Leber und Bauchspeicheldrüse
        \end{itemize}

    \subsection{Fütterungsfehler}
        \begin{itemize}
            \item \textbf{Einseitig} (nur FLeisch)
            \item \textbf{Knochen} (Verstopfungsgefahr)
            \item \textbf{Futterumstellung} zu schnell (Lösung: 7 Tage Übergang mit zunehmender/abnehmender Kombinationsmischung)
            \item\textbf{Milch} (Milchzucker schwer verdaulich)
            \item \textbf{Vitaminüberversorgung}
        \end{itemize}

    \subsection{Übergewicht}
        \begin{itemize}
            \item Belohnung und Leckerlis mitrechnen
            \item 1 Schweineohr = 20\% des täglichen Energiebedarfs
            \item \textbf{Idealgewicht}:
            \begin{itemize}
                \item Rippen gut tastbar
                \item dünne Fettauflage
                \item Taille sichtbar
            \end{itemize}
        \end{itemize}