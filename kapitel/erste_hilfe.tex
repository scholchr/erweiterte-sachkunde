%!TEX root = ../erweiterte_sachkunde.tex
\chapter{Erste Hilfe beim Hund}


\section{Erstmaßnahmen bei einem Unfall}
    \begin{itemize}
        \item \textbf{Absichern}: ruhig ansprechen und nähern, Schutz beteiligter Personen
        \item \textbf{Fang zubinden}, aber nicht bei:
        \begin{itemize}
             \item Bewusstlosigkeit
             \item Hitzschlag
             \item Atembeschweren
             \item Nasenbluten
             \item Erbrechen
             \item Verletzungen am Fang
             \item Herz-Kreislauf-Beschwerden
         \end{itemize}
        \item \textbf{Festbinden}: kurz anleinen
        \item \textbf{Vitalfunktionen überprüfen - TAPS}:
        \begin{itemize}
            \item \textbf{T}emperatur
            \item \textbf{A}tmung
            \item \textbf{P}uls
            \item \textbf{S}chleimhäute (Kreislauf)
        \end{itemize}
    \end{itemize}

    \subsection{Atemstillstand}
        \begin{itemize}
            \item Atemwege freilegen
            \item Zuge nach vorne ziehen
            \item Leichtes Überstrecken des Kopfes
            \item Beatmung wenn nach den o.g. Maßnahmen keine Spontanatmung einsetzt:
            \begin{itemize}
                \item Maul zudrücken (Zunge darf nicht heraushängen)
                \item Mund auf die Nasenlöcher des Hundes legen
                \item Kräftig und konstant Luft in die Nase blasen, bis sich der Brustkorb hebt
                \item Mehrere Male wiederholen
            \end{itemize}
        \end{itemize}

    \subsection{Pulskontrolle}
        Puls an der mittlere Innenseite des Oberschenkels mit dem Mittel- oder Zeigefinger kontrollieren. Dazu 15s zählen und den erfühlten Pulsschlag mit 4 multiplizieren

    \subsection{Schleimhäute}
        \begin{itemize}
            \item Farbe:
            \begin{itemize}
                \item Dunkelrot: hoher Blutdruck
                \item blassrosa: niedriger Blutdruck
                \item blau: Atemnot
            \end{itemize}
            \item Kapillarfüllungszeit:
            \begin{itemize}
                \item langsam (>2s): niedriger Blutdruck
                \item schnell (<2s): hoher Blutdruck
            \end{itemize}
        \end{itemize}

    \subsection{Normwerte}
    \begin{itemize}
        \item Temperatur: 38°-39°
        \item Atemfrequenz:
        \begin{itemize}
            \item große Hunde: 20-30/min
            \item kleine Hunde: 30-50/min
        \end{itemize}
        \item Schleimhäute: rosa
    \end{itemize}

    \subsection{Schockbehandlung}
        \begin{itemize}
            \item Schock erkennen: Absinken des Blutdrucks, Minderdurchblutung des Kopfes
            \item Symptome:
            \begin{itemize}
                \item Apathisches Verhalten
                \item Zittern
                \item Auskühlung
                \item Flache, schnelle Atmung
                \item Blasse, weisse Schleimhäute
                \item Flacher, schneller Puls
            \end{itemize}
            \item Schockbehandlung:
            \begin{itemize}
                \item Hund auf die Seite legen
                \item versuchen Hinterhand erhöht zu lagern
                \item warmhalten
                \item Kreislauf kontrollieren
                \item Bei Bewusstlosigkeit: Atemwege freimachen, ggf. beatmen
            \end{itemize}
        \end{itemize}

    \subsection{Bedrohliche Blutungen}