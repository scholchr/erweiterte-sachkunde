%!TEX root = ../erweiterte_sachkunde.tex
\chapter{Erste Hilfe beim Hund}


\section{Erstmaßnahmen bei einem Unfall}
    \begin{itemize}
        \item \textbf{Absichern}: ruhig ansprechen und nähern, Schutz beteiligter Personen
        \item \textbf{Fang zubinden}, aber nicht bei:
        \begin{itemize}
             \item Bewusstlosigkeit
             \item Hitzschlag
             \item Atembeschwerden
             \item Nasenbluten
             \item Erbrechen
             \item Verletzungen am Fang
             \item Herz-Kreislauf-Beschwerden
         \end{itemize}
        \item \textbf{Festbinden}: kurz anleinen
        \item \textbf{Vitalfunktionen überprüfen - TAPS}:
        \begin{itemize}
            \item \textbf{T}emperatur
            \item \textbf{A}tmung
            \item \textbf{P}uls
            \item \textbf{S}chleimhäute (Kreislauf)
        \end{itemize}
    \end{itemize}

    \subsection{Atemstillstand}
        \begin{itemize}
            \item Atemwege freilegen
            \item Zuge nach vorne ziehen
            \item Leichtes Überstrecken des Kopfes
            \item Beatmung wenn nach den o.g. Maßnahmen keine Spontanatmung einsetzt:
            \begin{itemize}
                \item Maul zudrücken (Zunge darf nicht heraushängen)
                \item Mund auf die Nasenlöcher des Hundes legen
                \item Kräftig und konstant Luft in die Nase blasen, bis sich der Brustkorb hebt
                \item Mehrere Male wiederholen
            \end{itemize}
        \end{itemize}

    \subsection{Pulskontrolle}
        Puls an der mittlere Innenseite des Oberschenkels mit dem Mittel- oder Zeigefinger kontrollieren. Dazu 15s zählen und den erfühlten Pulsschlag mit 4 multiplizieren

    \subsection{Schleimhäute}
        \begin{itemize}
            \item Farbe:
            \begin{itemize}
                \item Dunkelrot: hoher Blutdruck
                \item blassrosa: niedriger Blutdruck
                \item blau: Atemnot
            \end{itemize}
            \item Kapillarfüllungszeit:
            \begin{itemize}
                \item langsam (>2s): niedriger Blutdruck
                \item schnell (<2s): hoher Blutdruck
            \end{itemize}
        \end{itemize}

    \subsection{Normwerte}
    \begin{itemize}
        \item Temperatur: 38°-39°
        \item Atemfrequenz:
        \begin{itemize}
            \item große Hunde: 20-30/min
            \item kleine Hunde: 30-50/min
        \end{itemize}
        \item Pulsfrequenz:
        \begin{itemize}
            \item große Hunde: 70-80 Schläge/min
            \item kleine Hunde: 80-120 Schläge/min
        \end{itemize}
        \item Schleimhäute: rosa
    \end{itemize}

    \subsection{Schockbehandlung}
        \begin{itemize}
            \item Schock erkennen: Absinken des Blutdrucks, Minderdurchblutung des Kopfes
            \item Symptome:
            \begin{itemize}
                \item Apathisches Verhalten
                \item Zittern
                \item Auskühlung
                \item Flache, schnelle Atmung
                \item Blasse, weisse Schleimhäute
                \item Flacher, schneller Puls
            \end{itemize}
            \item Schockbehandlung:
            \begin{itemize}
                \item Hund auf die Seite legen
                \item versuchen Hinterhand erhöht zu lagern
                \item warm halten
                \item Kreislauf kontrollieren
                \item Bei Bewusstlosigkeit: Atemwege freimachen, ggf. beatmen
            \end{itemize}
        \end{itemize}

    \subsection{Bedrohliche Blutungen}
        \begin{itemize}
            \item Ohne Fremdkörper: Druckverband
            \item Mit Fremdkörper: steckenlassen, kein Druckverband
        \end{itemize}

    \subsection{Offene Brüche}
        \begin{itemize}
            \item Maulschlinge anlegen
            \item Bruch ruhig halten
            \item Hund auf unverletzte Seite legen
            \item Bruch mit sauberem Tuch abdecken
            \item Hund warm  halten
            \item Transport zum Tierarzt
        \end{itemize}


\section{Vergiftung}
    \subsection{Symptome}
        \begin{itemize}
            \item Starkes Erbrechen, starker Speichelfluß
            \item Reizung der Augen und Mundschleimhaut
            \item Zittern, Schwäche, Kreislaufkollaps
            \item Blut im Erbrochenen, Stuhl oder Urin
            \item Blaufärbung der Zunge
            \item später: starker Durchfall
        \end{itemize}

    \subsection{Maßnahmen}
        \begin{itemize}
            \item Aufbewahren der Probe des Giftes oder des Erbrochenen
            \item Erbrechen herbeiführen (Sauerkraut oder Wasser mit Senf anrühren und mit nadelloser Spritze ins Maul/Hals)
            \item Keinesfalls Maulschlaufe anlegen
            \item spätestens 2h nach Aufnahme zum Tierarzt
        \end{itemize}


\section{Magendrehung}
    Siehe \ref{ch:Anatomie:sec:Erkrankungen:subsec:Magendrehung}


\section{Verbrennungen}
    \subsection{Symptome}
        \begin{itemize}
            \item Haut:
            \begin{itemize}
                \item rot: Hitze
                \item weiß: Lauge
                \item braun: Säure
            \end{itemize}
            \item Fell an betroffener Stelle versengt oder fällt aus
            \item Haut zieht sich zusammen oder fühlt sich seifig/glitschig an
            \item Blasenbildung auf der Haut
        \end{itemize}

    \subsection{Maßnahmen}
        \begin{itemize}
            \item Eigene Hände schützen
            \item Wenn Haut intakt: mit Wasser ausspülen
            \item Wenn Haut geschädigt: Wunde mit sauberem Tuch abdecken
            \item Kühlende Kompresse
        \end{itemize}

\section{Hitzschlag}
    \subsection {Besonders betroffene Hunde}
        \begin{itemize}
            \item verkürzter Fang
            \item kurzer Nase
            \item langhaarige Rassen
            \item übergewichtige Hunde
            \item ältere Hunde
        \end{itemize}
    \subsection{Symptome}
        \begin{itemize}
            \item Starkes Hecheln
            \item sehr starker Speichelfluß
            \item Kreislaufkollaps
        \end{itemize}
    \subsection{Maßnahmen}
        \begin{itemize}
            \item Körpertemperatur senken
            \item Vorsichtig, aber intensiv mit Wasser kühlen
            \item reichlich zu trinken geben
            \item Beine massieren, um Kreislauf anzuregen
        \end{itemize}

\section{Erfrierung/Unterkühlung}
    \subsection{Symptome}
        \begin{itemize}
            \item Meistens an Schwanz, Ohrenspitzen und Fußballen
            \item Muskelsteifheit
            \item Zittern
            \item Niedrige Pulsfrequenz: unter 70/min
            \item niedrige Atemfrequenz: unter 10/min
            \item Haut löst sich in Schuppen ab, fühlt sich ledrig an
            \item Haarausfall, weiße Verfärbung des Fells
        \end{itemize}
    \subsection{Maßnahmen}
        \begin{itemize}
            \item Vorsichtig aufwärmen
            \item Bei Verwendung von Wärmflasche: in Tuch einwickeln
            \item Bei Verwendung von Fön: unterste Stufe
        \end{itemize}

\section{Insektenstiche}
    \begin{itemize}
        \item Insektenstiche im Maul- und Rachenraum können zu Schwellungen führen, die die Atmung erschweren oder gar unmöglich machen.
        \item Insektenstiche können zu allergischen Reaktionen führen
    \end{itemize}

    \subsection{Maßnahmen}
        \begin{itemize}
            \item Entfernen des Stachels mittels Pinzette
            \item Kühlen der Stelle mit Eisbeutel
            \item Tier beobachten, ggf. zum Tierarzt
        \end{itemize}

\section{Wunden und Verbände}
    \subsection{Wundarten}
        \begin{itemize}
            \item Schnittwunde: glatte Wundränder, blutet stark
            \item Stichwunde: kleine Öffnung, tiefe Wunden, äußerlich unscheinbar, hohes Infektionsrisiko
            \item Risswunde: zerfetzte Wundränder
            \item Rissquetschwunde: stumpfe Gewalteinwirkung (typisch bei Unfall)
            \item Bisswunden = Stich- oder Rissquetschwunden
            \item Schußwunde
            \item Schürfwunde
        \end{itemize}

\section{Wundversorgung}
    \begin{itemize}
        \item Kratzen und Lecken verhindern
        \item Ohne Verband: Immer wenn Wundflüssigkeit austritt
        \item Mit Verband: Große Wunden, keine Salben oder Wundsprays verwenden
        \item Bei tiefen Wunden: immer zum Tierarzt
    \end{itemize}

    \subsection{Notversorgung frischer Wunden}
        \begin{itemize}
            \item Fremdkörper nicht entfernen
            \item Nicht ausspülen
            \item Keimfreie Auflage
            \item Nichtsaugendes Druckpolster
            \item in Achter-Touren wickeln
            \item Druchpolster nicht öffnen
        \end{itemize}

    \subsection{Behandlung heilender Wunden}
        \begin{itemize}
            \item Wunde so ruhig wie möglich halten
            \item Nicht täglich säubern, wenn sie nicht infiziert ist
            \item Wunde trocken halten, höchstens Verbandswechsel
            \item Nach Möglichkeit Wunde nur beobachten
            \item Ausfluss nur abtupfen
            \item Wenn keine Heilung einsetzt: Tierarzt
        \end{itemize}