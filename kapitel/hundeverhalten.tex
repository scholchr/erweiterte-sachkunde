%!TEX root = ../erweiterte_sachkunde.tex
\chapter{Hundeverhalten}
\label{ch:Hundeverhalten}


\section{Hund und Wolf}
\label{ch:Hundeverhalten1:sec:Hund_und_Wolf}

    \subsection{Gemeinsamkeiten}
    \begin{itemize}
        \item Wolf als Stammvater des Hundes
        \item Obligat (zwingend) sozial (Rudeltiere)
        \item Randordnung
        \item Territorialverhalten
    \end{itemize}

    \subsection{Unterschiede}
    \begin{center}
        \begin{tabular}{c|l|l}
            & \textbf{Hund} & \textbf{Wolf} \\
            \hline
            Zusammenleben & Mensch/Hund & Familienverbund \\
            Nahrungserwerb & Dosenöffner Mensch & Nahrungserwerb lebenswichtig \\
            Spezialisierung & Spezialist & Allrounder \\
            Domestizierung & Domestiziert & Wildtier
        \end{tabular}
    \end{center}

    \subsection{Entstehung des Hundes}
    Zweistufentheorie:
    \begin{enumerate}
        \item Wölfe verlieren Scheu an Abfall
        \item Aktive Zähmung durch den Mensch
    \end{enumerate}


\section{Aggressionsverhalten}
\begin{itemize}
    \item Wertfreier Begriff
    \item sichert/verbessert Zugang zu Ressourcen (Futter/Fortpflanzung)
    \item Gruppe profitiert von Rangordnung
    \item Rangordnung wird nicht täglich neu geprüft
\end{itemize}


\section{Rangordnung}
Regeln für die \textbf{Rangordnung}:
\begin{itemize}
    \item Ernstkämpfe selten (Verletzungsgefahr)
    \item Ständige Kommunikation (optische/akustische/nasale Signale) erforderlich
    \item Die Summe der \textbf{Signale} entscheidet
\end{itemize}


\section{Kommunikation}

    \subsection{Optische Signale}
    \begin{tabular}{l|l}
        \textbf{Dominant} & \textbf{Unterwürfig} \\
        \hline
        Fixieren & Blick abwenden \\
        Ohren aufgerichtet & Ohren angelegt \\
        Gelenke durchgedrückt & Geduckte Haltung \\
        über Schnauze beissen & Mundwinkel lecken \\
        Schwanz hoch getragen & Schwanz eingezogen \\
        Maulspalte?? \\
    \end{tabular}

    \subsection{Signalspektrum}
        \begin{tabular}{l|l}
            \textbf{Schäferhund} & \textbf{Wolf} \\
            \hline
            12 mimische Signale & 60 mimische Signale \\
            6 Belllaute & Nur atonales Bellen \\
        \end{tabular}

    \subsection{Rassebesonderheiten}
        Mimik und Körpersprache werden durch Zucht beeinflusst \\
        $\rightarrow$ Missverständnisse vorprogrammiert z.Bsp.: Mimik bei Bulldogge, Haaresträuben bei Bobtail und Ohrenanlegen bei Beagle.

        \subsubsection{Rasseunterschiede Verhalten}
            \begin{tabular}{l|l}
                \textbf{Herdenschutzhunde} & \textbf{Schlittenhunde} \\
                \hline
                Territorialverhalten & Kaum Territorialverhalten \\
                Misstrauisch zu Fremden & Freundliche zu Fremden \\
                Kein Jagdverhalten & Jagdverhalten \\
                Geringer Bewegungsbedarf & Hoher Bewegungsbedarf \\
            \end{tabular}

    \subsection{Kommunikation Mensch-Hund}
        \begin{itemize}
            \item Optische Signale
            \begin{itemize}
                \item Körperhaltung
                \item Sichtzeichen
                \item Anstarren
            \end{itemize}
            \item Sprache
            \begin{itemize}
                \item Kurze eindeutige Kommandos
                \item Tonlage
            \end{itemize}
            \item Gerüche
        \end{itemize}


\section{Welpenentwicklung}

    \subsection{Einordnung}
        \begin{itemize}
            \item Hundeverhalten wird bestimmt durch \textbf{Erbanlagen} und \textbf{Lernen}
            \item Erbanlagen und Lernen beeinflussen sich gegenseitig.
        \end{itemize}

        \begin{itemize}
            \item Welpenentwicklung ist die Grundlage für ein normales Verhalten.
            \item Fehler / Versäumnisse sind schwer zu korrigieren.
        \end{itemize}

    \subsection{Phasen}
        \begin{enumerate}
            \item Neugeborenenphase (1. - 2. Woche)
            \item Übergangsphase (2. - 3. Woche)
            \item Sozialisationsphase (4. - 12. Woche)
        \end{enumerate}

    \subsection{Sozialisationsphase}
        \begin{itemize}
            \item Entwöhnung
            \item Angstäußerung bei Vereinzelung
            \item Gruppenspiele bzw. -aggression
            \item Erkundung der Umwelt
            \item Unsicherheit in unbekannten Situationen
        \end{itemize}

        Alles was erlebt wird, ist \glqq normal\grqq, daher:
        \begin{itemize}
            \item Autofahren (Boxentransport)
            \item Kontakt zu anderen Hunden und Rassen
            \item Menschen(gruppen)
            \item Geräusche
        \end{itemize}

    \subsection{Rasseunterschiede Welpenentwicklung}
        \begin{itemize}
            \item Golden Retriever: Umwelterkundung mit Geruchssinn
            \item Syberian Husky: früher koordiniertes Laufen
            \item Einzelne Bullterrierlinien: früh auftretende und gesteigerte Aggression
        \end{itemize}

    \subsection{Reizarme Aufzucht}
        \begin{itemize}
            \item Entwicklung des Stirnhirns beeinträchtigt
            \item Ängstlich-nervöses Verhalten
            \item Aggressivität
            \item Phobien
        \end{itemize}

    \subsection{Welpenabgabe}
        \begin{itemize}
            \item Fremdes \glqq Rudel\grqq
            \item Fremdes Territorium
            \item (Zu) Viel Aufmerksamkeit
            \item Zum ersten mal allein
        \end{itemize}


\section{Lernen}

    \subsection{Verarbeiten von Reizen}
        Die \textbf{Wahrnehmung} (Reize), \textbf{Erfahrung} und \textbf{Stimmung} sind Einflüsse auf das Gehirn und bestimmen das \textbf{aktive Verhalten}.
        Neben dem aktiven Verhalten gibt es noch eine \textbf{vegetative} (unterbewusste)\footnote{Wasser läuft im Maul zusammen.} und \textbf{hormonelle Reaktion}.

    \subsection{Lernen als biologischer Vorgang}
        \begin{itemize}
            \item Anpassung an veränderte Umwelt
            \item Bessere Möglichkeiten für Individuum (Futter u.a.)
            \item Sichtbar durch Verhaltensänderung
        \end{itemize}

    \subsection{Warum sollten Hunde lernen?}
        \begin{itemize}
            \item Vorteil Mensch: Erleichtert den umgang (Grundkommandos)
            \item Vorteil Hund: Beschäftigung und Abwechslung
            \item Lernen ist auch für ältere Hunde geeignet.
        \end{itemize}

    \subsection{Lernen als Grundlage}
        Lernen als Grundlage für:
        \begin{itemize}
            \item Orientierung des Hundes in der Umwelt
            \item Ausbildung von Hunden
            \item Verhaltenskorrektur
        \end{itemize}

    \subsection{Klassische Konditionierung nach Pavlov}
        \begin{itemize}
            \item Basis: Unbedingter Reiz (Futter) löst Reflex (Speicheln) aus.
            \item Koppelung: Reiz (Futter) wird mit Signal\footnote{bis dato: neutraler Reiz} mehrfach verknüpft
            \item Ergebnis: Das Signal allein löst nun einen Reflex aus. Der Hund ist nun konditioniert.
        \end{itemize}

    \subsection{Limbisches System}
        Im limbische (Belohnungs)System(LB) ist die \textbf{Motivationszone} im Gehirn und organisiert das zielorientierte Verhalten. Es reguliert:
        \begin{itemize}
            \item Angst
            \item Freude
            \item Trauer
            \item Aggression
            \item Motivation
            \item Sexualverhalten
            \item Brutpflege
        \end{itemize}

        \subsubsection{Gehirnaktivität}
            \begin{itemize}
                \item LB: Als Reaktion eine Belohnung findet im Gehirn eine Aktivitätenerhöhung statt. Der Hund ist motiviert.
                \item klassische Konditionierung: Die Motivation/Erwartungshaltung erfolgt hierbei nach dem Signal. Die Belohnung hat wenig Wirkung auf das Gehirn.
                \item Wird ein Signal ohne anschliessende Belohnung gegeben ist der Hund sogar frustiert.
            \end{itemize}

    \subsection{Operante Konditionierung}
        Verhalten tritt \textbf{öfter} auf:
        \begin{itemize}
            \item bei Erfolg (Verstärkung)
            \item wenn Unangenehmes aufhört
        \end{itemize}

        Verhalten tritt \textbf{seltener} auf:
        \begin{itemize}
            \item wenn es keinen Erfolg hat
            \item wenn unangenehmes erfolgt
        \end{itemize}

        Beispiele: \\
        \begin{tabular}{l|l}
            \textbf{Reiz} & Menschen essen \\
            \textbf{Verhalten} & Hund bettelt \\
            \textbf{Konsequenz} & Hund erhält etwas \\
            \textbf{Zukünftiges Verhalten} & Hund wird zukünftig betteln \\
        \end{tabular}

        \begin{tabular}{l|l}
            \textbf{Reiz} & Menschen \\
            \textbf{Verhalten} & Hund springt Mensch an \\
            \textbf{Konsequenz} & Hund wird ignoriert \\
            \textbf{Zukünftiges Verhalten} & Hund wird zukünftig nicht mehr anspringen \\
        \end{tabular}

        \subsubsection{Sonderform: selbstbelohnendes Verhalten}
            Verhalten folgt auf Reiz weitgehend unabhängig von Konsequenz:
            \begin{itemize}
                \item Jagdverhalten
                \item Sexualverhalten
                \item Brutpflegeverhalten
            \end{itemize}


\section{Grundlagen der Hundeausbildung}

\begin{itemize}
    \item Timing
    \begin{itemize}
        \item Verknüpfungszeit extrem kurz (\textbf{1 sec})
    \end{itemize}
    \item Reizintensität / Belohnung
        \begin{itemize}
            \item Lob, Zuwendung
            \item Leckerli: sollte attraktiv und leicht abschluckbar sein (sonst: Konzentrationsverlust)
            \item Wirkt nur, wenn sie etwas Besonderes ist.
            \item Anwendung beim ERlenen eines neuen Verhaltens:
                \begin{itemize}
                    \item Zu Beginn: Jedes mal!
                    \item Später: Nicht jede Aktion belohnen
                    \item \glqq Unvorhersehbar\grqq belohnen
                \end{itemize}
        \end{itemize}
    \item Konsequenz
        \begin{itemize}
            \item Auftrainiertes Verhalten wird erst nach \textbf{1000 Wiederholungen} sicher gezeigt
        \end{itemize}
\end{itemize}

    \subsection{Erziehen durch Strafe?}
        \begin{itemize}
            \item Falsches Timing extrem schädlich
            \item Unerwünschtes Verknüpfen (Strafreiz + Umgebung) \\
                  Beispiel Stachelhalsband: Hund verknüpft Schmerz mit anderem Hund anstatt mit dem \glqq Nach vorne gehen\grqq
            \item Strafe = Stress (Lernfähigkeit beeinträchtigt)
            \item Vertrauensverlust
            \item \textbf{Strafe ändert keine Emotionen!}
            \item \textbf{Kein Erziehen durch Schmerz + Strafe!}
        \end{itemize}

        \subsubsection{Erlaubte Korrekturen}
            \begin{itemize}
                \item Ignorieren
                \item Schnauzengriff
                \item Wegschicken
                \item Stimmsignale: \glqq Nein!\grqq (Nicht Lautstärke)
            \end{itemize}

    \subsection{Lernen klappt nicht - Wieso?}
        \begin{itemize}
            \item Bedrohung durch Besitzer / Trainer
            \item Stress von Besitzer / Trainer
            \item Zu hohe Anforderung
            \item Unsicherheit
        \end{itemize}

        \subsubsection{Anzeichen von Stress und Angst}
            \begin{itemize}
                \item Häufige Beschwichtigungssignale
                \item Geduckte Haltung
                \item Anspannung
                \item Ohren angelegt, Schwanz eingezogen
            \end{itemize}

    \subsection{Es klappt immer noch nicht}
        \begin{itemize}
            \item Zu lange Übungseinheiten
            \item Ablenkung durch Außenreize
            \item Ortverknüpfung
            \item \glqq Geräusche\grqq statt Kommandos
        \end{itemize}

        \subsubsection{Lösung}
        \begin{itemize}
            \item \textbf{Gelassen bleiben!}
            \item Außenreize ausschalten
            \item Übungsort wechseln
            \item Klare Kommandos
            \item Niedrigeres Trainingsziel
            \item Abbruch ist keine Schande
        \end{itemize}

    \subsection{Belohnung durch Clickern}
        \begin{itemize}
            \item Sekundärer (erlernter) Verstärker
            \item exaktes Timing möglich
            \item Auffälliges Signal
            \item Muss zuerst gelernt werden (klassische Konditionierung)
            \item Clickern + Belohnung immer in Kombination
            \item Clickern ist kein Befehl
        \end{itemize}

    \subsection{Bei allen neuen Trainingsmethoden}
        \begin{itemize}
            \item Immer überprüfen, ob die Gesetze der Lernbiologie beachtet werden
            \item Wundermethoden, die das nicht tun, sind unseriös!
        \end{itemize}