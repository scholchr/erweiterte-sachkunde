%!TEX root = ../erweiterte_sachkunde.tex
\chapter{Hundeverhalten}
\label{ch:Hundeverhalten}

\section{Hund und Wolf}
\label{ch:Hundeverhalten1:sec:Hund_und_Wolf}

    \subsection{Gemeinsamkeiten}
    \begin{itemize}
        \item Wolf als Stammvater des Hundes
        \item Obligat (zwingend) sozial (Rudeltiere)
        \item Randordnung
        \item Territorialverhalten
    \end{itemize}

    \subsection{Unterschiede}
    \begin{center}
        \begin{tabular}{c|l|l}
            & \textbf{Hund} & \textbf{Wolf} \\
            \hline
            Zusammenleben & Mensch/Hund & Familienverbund \\
            Nahrungserwerb & Dosenöffner Mensch & Nahrungserwerb lebenswichtig \\
            Spezialisierung & Spezialist & Allrounder \\
            Domestizierung & Domestiziert & Wildtier
        \end{tabular}
    \end{center}

    \subsection{Entstehung des Hundes}
    Zweistufentheorie:
    \begin{enumerate}
        \item Wölfe verlieren Scheu an Abfall
        \item Aktive Zähmung durch den Mensch
    \end{enumerate}

\section{Aggressionsverhalten}
\begin{itemize}
    \item Wertfreier Begriff
    \item sichert/verbessert Zugang zu Ressourcen (Futter/Fortpflanzung)
    \item Gruppe profitiert von Rangordnung
    \item Rangordnung wird nicht täglich neu geprüft
\end{itemize}

\section{Rangordnung}
Regeln für die \textbf{Rangordnung}:
\begin{itemize}
    \item Ernstkämpfe selten (Verletzungsgefahr)
    \item Ständige Kommunikation (optische/akustische/nasale Signale) erforderlich
    \item Die Summe der \textbf{Signale} entscheidet
\end{itemize}

\section{Kommunikation}

    \subsection{Optische Signale}
    \begin{tabular}{l|l}
        \textbf{Dominant} & \textbf{Unterwürfig} \\
        \hline
        Fixieren & Blick abwenden \\
        Ohren aufgerichtet & Ohren angelegt \\
        Gelenke durchgedrückt & Geduckte Haltung \\
        über Schnauze beissen & Mundwinkel lecken \\
        Schwanz hoch getragen & Schwanz eingezogen \\
        Maulspalte?? \\
    \end{tabular}

    \subsection{Signalspektrum}
        \begin{tabular}{l|l}
            \textbf{Schäferhund} & \textbf{Wolf} \\
            \hline
            12 mimische Signale & 60 mimische Signale \\
            6 Belllaute & Nur atonales Bellen \\
        \end{tabular}

    \subsection{Rassebesonderheiten}
        Mimik und Körpersprache werden durch Zucht beeinflusst \\
        $\rightarrow$ Missverständnisse vorprogrammiert z.Bsp.: Mimik bei Bulldogge, Haaresträuben bei Bobtail und Ohrenanlegen bei Beagle.

        \subsubsection{Rasseunterschiede Verhalten}
            \begin{tabular}{l|l}
                \textbf{Herdenschutzhunde} & \textbf{Schlittenhunde} \\
                \hline
                Territorialverhalten & Kaum Territorialverhalten \\
                Misstrauisch zu Fremden & Freundliche zu Fremden \\
                Kein Jagdverhalten & Jagdverhalten \\
                Geringer Bewegungsbedarf & Hoher Bewegungsbedarf \\
            \end{tabular}

    \subsection{Kommunikation Mensch-Hund}
        \begin{itemize}
            \item Optische Signale
            \begin{itemize}
                \item Körperhaltung
                \item Sichtzeichen
                \item Anstarren
            \end{itemize}
            \item Sprache
            \begin{itemize}
                \item Kurze eindeutige Kommandos
                \item Tonlage
            \end{itemize}
            \item Gerüche
        \end{itemize}